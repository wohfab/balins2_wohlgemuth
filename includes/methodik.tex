\chapter{Methodik}
\label{chap:Methodik}
\pagestyle{plain}

Vergleich der Phrasierung im Chorsatz (Noten) zu (flacher) syntaktischer Phrasierung (z.B. aus NLTK Chunk Parser???).

Im folgenden Abschnitt werde ich die zwei Stücke getrennt nach Melodie und Text auf Ihre Phrasierung untersuchen. Dabei starte ich mit der Melodie, die sich aus den Noten mithilfe von Pausenzeichen in gesungene Phrasen einteilen lässt.

\tiny 

\subsection*{Strophe 1}

\Tree [.Phrase-1 \qroof{A light in the room}. ] \Tree [.Phrase-2 \qroof{It was you who was standing there}. ] \Tree [.Phrase-3 \qroof{Tried, it was true}. ] \Tree [.Phrase-4 \qroof{As your glance met my stare}. ] 
\Tree [.Phrase-1 \qroof{But your heart drifted off}. ] \Tree [.Phrase-2 \qroof{Like the land split by sea}. ] \Tree [.Phrase-3 \qroof{I tried to go, to follow}. ] \Tree [.Phrase-4 \qroof{To kneel down at your feet}. ] 

\subsection*{Refrain}

\Tree [.Phrase-1 \qroof{I'll run}. ] \Tree [.Phrase-2 \qroof{I'll run}. ] \Tree [.Phrase-3 \qroof{I'll run}. ] \Tree [.Phrase-4 \qroof{run to you}. ] 
\Tree [.Phrase-1 \qroof{I'll run}. ] \Tree [.Phrase-2 \qroof{I'll run}. ] \Tree [.Phrase-3 \qroof{I'll run}. ] \Tree [.Phrase-4 \qroof{run to you}. ] 
