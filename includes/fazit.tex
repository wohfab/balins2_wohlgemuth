\chapter{Fazit und Ausblick}
\label{chap:Fazit und Ausblick}

% Ergebnisse zusammenfassen und bewerten - Beantworten der Fragen aus der Einleitung - Ausblick / Offene Fragen / Angrenzende Themengebiete

% KEINE neuen Erkenntnisse / Thesen

Die Untersuchungen zeigen, dass es wenige Unterschiede ....

Eine umfangreichere Studie könnte diese Ergebnisse bekräftigen. Aufgrund der manuellen Vorgehensweise, ist die Ausarbeitung in dieser Hausarbeit auf zwei Chor-Stücke beschränkt. Ließen sich Teile der Methodik automatisieren und zum Beispiel über computerlinguistische Analyse ersetzen, könnte ein weit größerer Datensatz genutzt werden, um die Signifikanz der Ergebnisse zu erhöhen.

Situationsbedingt habe ich bei der Bearbeitung der vorliegenden Texte auf die Form der Aufnahme von Sprecher:innen und Sänger:innen verzichtet und mich lediglich an der theoretischen Ausarbeitung von Phrasierung orientiert. Die Ergebnisse könnten in einem weiteren Schritt durch entsprechende Aufnahmen der Texte weitergehend bestätigt werden. Lassen die äußeren Rahmenbedingungen dies zu, so könnte man auch rein praktisch vorgehen und im Zweifelsfall realitätsnähere Ergebnisse erzielen.

Die Notation betreffend, gibt es einiges an Potential. Wie in \cite{liberman1975intonational}, \cite{lerdahl1983generative} und auch \cite{hayes1996role}, bietet sich für eine tiefergehende Analyse der Daten auch eine umfänglichere Notation an, die Melodie- und Text-Teil kombiniert darstellt. Da mir die benötigten Daten für eine solche Darstellung leider nicht zur Verfügung standen und ich als Noten-Grundlage lediglich PDF-Dateien vorliegen hatte, kam eine Umwandlung im Rahmen der Arbeit nicht in Frage.