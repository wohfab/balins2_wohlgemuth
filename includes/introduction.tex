\chapter{Introduction}
\label{chap:Introduction}

\begin{quote}
\textit{''Music is the universal language of mankind.''}
\\--- Henry Wadsworth Longfellow

\textit{''Wer hört auf die Worte, wo Töne siegen!''} 
\\--- Richard Strauss, Capriccio (scene 3)
\end{quote}


\vspace{1cm}

% Zielsetzung - Problemstellung - Eingrenzung/Abgrenzung des Themas (begründet) - Aufbau - Roter Faden

When researching literature for a review on the topic of linguistics and singing, next to Johan Sundberg's singing formant research, there are many small and isolated publications on many research topics, only tangential to the interface between linguistics and music.

The following module paper works towards possible approaches to establish an interface between linguistics and music, by reviewing the existing research.
To find said connection, I reviewed research papers and book chapters, dealing with various fields of research, including aspects of music, linguistics, and cognition, all with singing as the common denominator. The aim for this paper is to give a broad and shallow overview of related research to the linguistics and singing interface.

During the research for this paper, it became clear, that many different fields of research tap into the singing domain in various ways. To keep the focus on  linguistics, I will structure this paper, following the main branches of linguistics, give short introductions to the branches, and point out their relations to singing. Starting with  \nameref{Phonetics and Phonology} in \autoref{Phonetics and Phonology}, I will include obvious similarities in sound production, but also my findings of importance of at least implicit knowledge of the IPA chart, and differences in singing styles, that can be evaluated by phonetic research. Then I will proceed with the \nameref{Morphology and Syntax} in \autoref{Morphology and Syntax}, with \nameref{Semantics and Pragmatics} in \autoref{Semantics and Pragmatics}, and finish the overview with \nameref{Language Learning and Multi-Lingualism} in \autoref{Language Learning and Multi-Lingualism}.  At some of those sections, I will point out a few interesting aspects of research, I expected to, but was not able to find during my research.

As many publications tap into multiple of linguistics' branches, I will recapitulate the findings in a \nameref{chap:Conclusion} in \autoref{chap:Conclusion}.