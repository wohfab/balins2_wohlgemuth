\chapter{Conclusion}
\label{chap:Conclusion}

% Ergebnisse zusammenfassen und bewerten - Beantworten der Fragen aus der Einleitung - Ausblick / Offene Fragen / Angrenzende Themengebiete
% KEINE neuen Erkenntnisse / Thesen

In this section, I will summarize the findings of my literature review by the already introduced branches of linguistics. After summarizing the findings, I will summarize the gaps in findings as well, to show interesting subjects, still to be researched.




In the \nameref{Phonetics and Phonology} branch in \autoref{Phonetics and Phonology}, I was able to find most of the links to singing. Prosodic links, which were no surprise, since singing is linked to musical melody, the ability to voluntarily change larynx posture, which gives us Sundberg's \textit{Singing Formant}, and strong links to the IPA chart, being crucial to choral singing, especially in different languages, so that everyone in the choir articulates the same phonemes. I also found a publication, talking about the positive effect of singing abilities on pitch variation detection, and the fact, that the impossibility of vowel detection above an F5 note, that lies around 700 Hz, can be visualized via measurements of the formants. As I mentioned before, I was looking forward to find an answer to the question of melody in tonal languages, which I also described in this section. At last, and as one of the more direct links between linguistics and singing, I showed the results of Sundberg's analysis of different singing styles and their link to speaking phonation modes.

For the \nameref{Morphology and Syntax} \autoref{Morphology and Syntax}, I introduced studies on a \textit{generative theory of tonal music}, saying musical hierarchical structures are similar to those used in syntactic analyses of natural languages.

The conveying of meaning via melody and words, covered in the \nameref{Semantics and Pragmatics} \autoref{Semantics and Pragmatics}, was no surprise to me as a choral singer.Being able to replicate a singer's/speaker's facial expression just by listening to the sound of the words, is a task, a choir singer faces every time, singing with their colleagues. The method of exactly replicating each other's vowel sounds in unison choral singing is called \textit{choral blend} or \textit{blending} \cite{goodwin1980acoustical}.

The most surprising chapter for me was the \nameref{Language Learning and Multi-Lingualism} \autoref{Language Learning and Multi-Lingualism}. I was able to find publications about musical abilities enhancing parts of language learning, and the other way round, about the ability to speak multiple languages, being a positive influence on music abilities.






When researching for this paper, I thought about interesting subjects, to combine music and linguistics via the singing domain. Before reading a publication, I wrote down my thoughts of how to link linguistic research to singing in the given topic, to see, if the used methodologies and the found results match my expectations. As shown above, they did match my expectations in some publications. In others, I had thought of different approaches, and sometimes even of approaches to link singing to linguistics, that I could not find a publication for. In the following paragraphs, I will recapitulate the gaps in findings.

Stuttering and accents in speaking are well researched. The research of stuttering in singing, seen from a linguistic point of view, is not. Differences from speaking to singing, concerning the subject of stuttering, are almost exclusively researched on the cognitive side of language. Linguistic approaches, being communicational for example, are not well researched. One cited publication mentioned the communication aspect, present in speaking, but missing in singing. Analysing the differences in speech between speaking and singing, and finding vocal characteristics, that reduce stuttering, could lead to effective speech therapy with a singing approach, to reduce stuttering in speaking as well.

The last branch I introduced in my paper, was computational linguistics. As for my background being in computational linguistics, my interest in research in this topic is apparent. Two of the main subjects of computational linguistics, that are present outside academia and already well established in the commercial world, are speech recognition and speech synthesis. Both of those topics are well researched and mostly use speech data to work. Unsurprisingly, the findings of singing data in computational speech recognition and synthesis were rare. While I found one study of speech recognition, using singing data, the researchers only used a single approach to train their phoneme model for recognition. They trained on speech data and afterwards added singing characteristics to the data. A direct approach of using singing data to train said phoneme model would require a larger dataset, but would be quite interesting to test.

Speech synthesis, the programmatic creation of speech data via a computer, is the other main subject of computational linguistics, from which I expected to find a link to singing. As, on the one hand, the use of speech synthesis is already common in multiple domains, and on the other hand, auto-tune, as a matter of computational pitch modification, is used widely in popular music, I expected to find publications on synthetically creating singing parts of music.







Mapping the existing research of the linguistic and singing interface, gave me a good overview about topics, that lack depth in research. As I depicted in this conclusion, there are multiple open research questions and different approaches, to be tested, the computational linguistics research approaches being the ones, I am most interested in, to review again in the future.