\chapter{Einleitung}
\label{chap:Einleitung}

% schriftliche Arbeit im Umfang von 10-15 Seiten

\begin{quote}
\textit{''Music is the universal language of mankind.''}
\\--- Henry Wadsworth Longfellow

\textit{''Wer hört auf die Worte, wo Töne siegen!''} 
\\--- Richard Strauss, Capriccio (scene 3)
\end{quote}

\vspace{1cm}

Die vorliegende Hausarbeit wurde erstellt im Rahmen der Modulabschlussprüfung des Vertiefungsmoduls im Bachelorstudium Linguistik mit Profil Sprache. Sie befasst sich mit Unterschieden zwischen gesungener und gesprochener Sprache und konzentriert sich in diesem Themenfeld auf die Phrasierung.

Durch Mitgliedschaften in diversen kleinen und großen Vokal-Ensembles, bin ich schon seit einiger Zeit in den Kreisen der Stimm-fokussierten Musik zu Hause. Aus eben selben Kreisen heraus ist die Motivation für diese Hausarbeit entstanden, die Sprache und Musik zu verbinden versucht.

In \autoref{sec:Definition: Phrasierung} werde ich definieren, was unter Phrasierung zu verstehen ist.
Im Anschluss werde ich in \autoref{chap:Datensatz} den \nameref{chap:Datensatz} beschreiben. Dieser besteht aus zwei englischsprachigen Chor-Arrangements. Ich werde die Struktur der Stücke umreißen und sowohl die Liedtexte als auch die relevanten Melodien herausarbeiten.
Nach Erläuterung des Datensatzes werde ich in \autoref{chap:Methodik} erläutern, mit welcher Methodik ich die Texte und Melodien analysieren werde. Hier greife ich auf theoretische Modelle und praktische Anwendungen der Computerlinguistik zurück.
Die \nameref{chap:Ergebnisse} werde ich in \autoref{chap:Ergebnisse} aufzeigen und ausarbeiten.
Anschließend werden die Ergebnisse in \autoref{chap:Diskussion} diskutiert. Hier zeige ich, welche Erkentnisse aus meinen Untersuchungen gewonnen werden konnten und welchen Wert sie innerhalb der Forschung einnehmen können.
In \autoref{chap:Fazit und Ausblick} werde ich ein Fazit aus der Arbeit ziehen und einen Ausblick geben, inwieweit die Forschungen ausgeweitet werden können. Außerdem werde ich darauf eingehen, welche Probleme während der Arbeit aufkamen und wie ich die Arbeit hätte verbessern können.

% Welches Ziel verfolge ich?

% Was ist das Problem?

% Eingrenzung und Abgrenzung des Themas

% Aufbau erklärt

% Roter Faden einlieiten

% Forschungsinteresse begründen?




\section{Definition: Phrasierung}
\label{sec:Definition: Phrasierung}

Bei der Phrasierung handelt es sich um eine...